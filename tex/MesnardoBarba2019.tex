\documentclass[10pt,journal,compsoc]{IEEEtran}

\usepackage{algorithmic}
\usepackage{array}
 \usepackage[nocompress]{cite}
 \usepackage{color}
\usepackage{listings}
\usepackage[caption=false,font=normalsize,labelfont=sf,textfont=sf]{subfig}
\usepackage{stfloats}
\usepackage{url}


\usepackage[pdftex]{graphicx}
\graphicspath{{./figs}}
\DeclareGraphicsExtensions{.pdf,.jpeg,.png}


% *** MATH PACKAGES ***
\usepackage{amsmath}
\interdisplaylinepenalty=2500

% correct bad hyphenation here
\hyphenation{op-tical net-works semi-conduc-tor}


\begin{document}

% title
\title{Reproducible workflow on a public cloud for computational fluid dynamics}
% author names and affiliations
\author{Olivier Mesnard, Lorena A. Barba
\IEEEcompsocitemizethanks{\IEEEcompsocthanksitem Mechanical and Aerospace Engineering,
the George Washington University, Washington, DC 20052.\protect\\
% note need leading \protect in front of \\ to get a newline within \thanks as
% \\ is fragile and will error, could use \hfil\break instead.
E-mail: mesnardog@gwu.edu
\IEEEcompsocthanksitem Email: labarba@gwu.edu}% <-this % stops an unwanted space
%\thanks{Manuscript submitted 2019}
}

\IEEEtitleabstractindextext{%
\begin{abstract}
In a new effort to make our research transparent and reproducible by others, we have developed a workflow to run computational studies on a public cloud. It uses Docker containers to create an image of the application software stack. We also adopt several tools that facilitate creating and managing virtual machines with compute nodes and submitting jobs to these nodes. The configuration files for these tools are part of an expanded "reproducibility package" that includes workflow definitions for cloud computing, in addition to input files and instructions. This facilitates re-creating the cloud environment to re-un the simulations under the same conditions.
\end{abstract}
}

% make the title area
\maketitle


% abstract


% start the article



% \section*{Acknowledgment}

%\bibliographystyle{IEEEtran}
% argument is your BibTeX string definitions and bibliography database(s)
%\bibliography{IEEEabrv,../bib/paper}
%
% <OR> manually copy in the resultant .bbl file
%\begin{thebibliography}{1}
%\bibitem
%\end{thebibliography}

\end{document}


